\chapter{绪论}
\label{cha:intro}

\section{引言}

响应理论是实验探测和器件设计的基础。材料的响应指的是在外场的作用下,材料的可观测物理量(用算符$\hat{O}$表示)发生变化的行为。这里的外场一般指的是电场(用$E$表示)或者磁场(用$B$表示)。由于在常见光源中,电磁波中的磁场对材料并没有太大的影响, 因此光场一般可以看成交变电场。由于在施加外场的时候,外场的方向,时间依赖关系,空间依赖关系均可以变化,实验上对材料的探测就有了很多的可变参数,可以探测材料的多种不同性质。实验上的观测物理量则更加多样。传统的可观测物理量有电流电导(霍尔效应),磁矩(量子振荡),极化(铁电回线)等等。近年来,人们开始尝试探索电子的新的自由度,由此诞生了一些新的物理可观测量,例如自旋流(自旋霍尔效应),谷极化流(谷霍尔效应)等。这些手段极大地丰富了可能凝聚态实验的手段,并且对工程应用也有极大的价值。


由于实验探测的目的主要是了解材料本身的性质,因此实验施加的外场往往较小。在这样的背景下,可观测量偏离平衡态的大小($\delta\langle\hat{O}\rangle$)经常对外场场强进行小量展开。除非对称性禁止,小量展开的最低阶往往是线性的($\delta\langle\hat{O}\rangle \propto E$或者$B$),这样的响应称之为线性响应。随着实验技术的发展和实验目的的变化,人们有时也对研究的样品材料施加强场,以期获得高阶的响应,这样的响应统称为高阶响应。其中,二阶响应是人们研究最多的对象。

一般而言,由于电子之间存在着交换关联作用,计算一个系统对外场的响应尤其是高阶响应是一件比较困难的事情。幸运的是,对于非强关联体系,独立粒子近似就能够较好地预测实验结果。因此,本文的理论分析基本基于独立粒子近似,而计算手段则采用基于密度泛函理论的能带论。

在独立粒子近似下,计算材料的响应有很多中手段,最流行的两种手段是格林函数和准经典运动方程。这两种手段在输运理论中都取得了巨大的成功。格林函数,尤其是非平衡格林函数,应用面更加广泛。而准经典运动计算更加简单更加直观。下面本文将对准经典运动方程做一个简要介绍,然后列举出一些本文将会探讨的响应现象。

\section{准经典运动方程}

准经典运动方程是基于波包的动力学。波包指的是用同一条能带上的多个布洛赫波叠加得到的量子态,波包不具有确定的晶格动量,但是具有良好定义的位置算符的本征值。具体来说,波包$|W\rangle$定义为
\begin{equation}
|W\rangle=\int d\bk w(k, t)|\phi_n(\bk)\rangle.
\end{equation}
其中,$\phi_n$是第$n$条带上的布洛赫波函数,而$w(\bk,t)$则是叠加的权重。一般而言这个权重是局域在某一个晶格动量上的
\begin{equation}
\bk_c = \int dk \bk w(\bk,t).
\end{equation}
此时,这个波包的位置是具有良好定义的期望值的
\begin{equation}
\boldsymbol{r}_c = \left. -\frac{\partial}{\partial \bk} \arg w(\bk, t) + \berry_n(\bk)\right|_{\bk=\bk_c}.
\end{equation}
这里的$\berry_n(\bk)=i \langle u_n(\bk)|\nabla_{\bk}|u_n(\bk) \rangle$被称为贝里联络,$u_n(\bk)$则是布洛赫波的周期性部分。
\section{位移电流}
\section{布洛赫电子的磁性质}
\subsection{量子振荡}