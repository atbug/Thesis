\chapter{绪论}
\label{cha:intro}

\section{引言}

响应理论是实验探测和器件设计的基础。材料的响应指的是在外场的作用下,材料的可观测物理量(用算符$\hat{O}$表示)发生变化的行为。这里的外场一般指的是电场(用$E$表示)或者磁场(用$B$表示)。由于在常见光源中,电磁波中的磁场对材料并没有太大的影响, 因此光场一般可以看成交变电场。由于在施加外场的时候,外场的方向,时间依赖关系,空间依赖关系均可以变化,实验上对材料的探测就有了很多的可变参数,可以探测材料的多种不同性质。实验上的观测物理量则更加多样。传统的可观测物理量有电流电导(霍尔效应),磁矩(量子振荡),极化(铁电回线)等等。近年来,人们开始尝试探索电子的新的自由度,由此诞生了一些新的物理可观测量,例如自旋流(自旋霍尔效应),谷极化流(谷霍尔效应)等。这些手段极大地丰富了可能凝聚态实验的手段,并且对工程应用也有极大的价值。


由于实验探测的目的主要是了解材料本身的性质,因此实验施加的外场往往较小。在这样的背景下,可观测量偏离平衡态的大小($\delta\langle\hat{O}\rangle$)经常对外场场强进行小量展开。除非对称性禁止,小量展开的最低阶往往是线性的($\delta\langle\hat{O}\rangle \propto E$或者$B$),这样的响应称之为线性响应。随着实验技术的发展和实验目的的变化,人们有时也对研究的样品材料施加强场,以期获得高阶的响应,这样的响应统称为高阶响应。其中,二阶响应是人们研究最多的对象。

一般而言,由于电子之间存在着交换关联作用,计算一个系统对外场的响应尤其是高阶响应是一件比较困难的事情。幸运的是,对于非强关联体系,独立粒子近似就能够较好地预测实验结果。因此,本文的理论分析基本基于独立粒子近似,而计算手段则采用基于密度泛函理论的能带论。

在独立粒子近似下,计算材料的响应有很多中手段,最流行的两种手段是格林函数和准经典运动方程。这两种手段在输运理论中都取得了巨大的成功。格林函数,尤其是非平衡格林函数,应用面更加广泛。而准经典运动计算更加简单更加直观。下面本文将对准经典运动方程做一个简要介绍,然后列举出一些本文将会探讨的响应现象。

\section{准经典运动方程}

准经典运动方程是基于波包的动力学。波包指的是用同一条能带上的多个布洛赫波叠加得到的量子态,波包不具有确定的晶格动量,但是具有良好定义的位置算符的本征值。具体来说,波包$|W\rangle$定义为
\begin{equation}
|W\rangle=\int d\bk w(k, t)|\phi_n(\bk)\rangle.
\end{equation}
其中,$\phi_n$是第$n$条带上的布洛赫波函数,而$w(\bk,t)$则是叠加的权重。一般而言这个权重是局域在某一个晶格动量上的
\begin{equation}
\bk_c = \int dk \bk w(\bk,t).
\end{equation}
此时,这个波包的位置是具有良好定义的期望值的
\begin{equation}
\boldsymbol{r}_c = \left. -\frac{\partial}{\partial \bk} \arg w(\bk, t) + \berry_n(\bk)\right|_{\bk=\bk_c}.\label{semi-rc}
\end{equation}
这里的$\boldsymbol{\berry}_n(\bk)=i \langle u_n(\bk)|\nabla_{\bk}|u_n(\bk) \rangle$被称为贝里联络,$u_n(\bk)$则是布洛赫波的周期性部分。

$\bk_c$ 和 $\br_c$ 是波包的重要性质。在经典的牛顿力学理论中,这两个量分别对应粒子的动量和位置。由于不确定原理,这两个量在量子力学中仅仅是期望值。 在外电场和磁场很小的情况下,$\bk_c$ 和 $\br_c$的运动方程是(下标$c$省略)
\begin{align}
	\frac{d\br}{dt} &= \frac{\partial \epsilon_M(\bk)}{\hbar\partial \bk}-\frac{d\bk}{dt}\times\boldsymbol{\Omega}(\bk)\nonumber\\
	\hbar\frac{d\bk}{dt} &= -e\boldsymbol{E}-e\frac{d\br}{dt}\times\boldsymbol{E}.\label{semi-eqn}
\end{align}
这样的方程和经典的牛顿运动方程非常相似,但是存在着几项修正,下面我们一项一项分析其修正。

第一个修正来自于轨道磁矩,和牛顿力学中的粒子不同,波包并没有确定的位置,波包的波函数在实空间里面是延展的。因此波包可以围绕这自己的中心旋转,这样的旋转会带来不为0的轨道磁矩,在磁场中,轨道磁矩会贡献额外的能量,因此第一项修正为$\epsilon_M(\bk)=\epsilon(\bk)-\boldsymbol{m}(\bk)\cdot\boldsymbol{B}$。这里的$\epsilon(\bk)$是没有外场下的布洛赫波的能量,而轨道磁矩的计算表达式为
\begin{equation}
\boldsymbol{m}(\bk) = -i\frac{e}{2\hbar}\langle\nabla_{\bk}u|\times [\hat{H}(\bk)-\epsilon(\bk)]|\nabla_{\bk}u\rangle.
\end{equation}
可以看到轨道磁矩并不和波包的具体形式有关,只和体的波函数相关。这给计算带来的巨大的便利。

第二项修正来自于贝里曲率,这又是一项超越经典力学的修正项。经典力学中粒子的运动是用实数描述的,而量子力学中的波函数则必须使用复数来描述,复数和实数的巨大差异就来源于复数存在相位。贝里曲率则来源于波包运动过程中产生的几何相位。贝里曲率的定义是
\begin{equation}
\boldsymbol{\Omega}(\bk) = \nabla\times\berry(\bk).
\end{equation}
它贡献了式\ref{semi-eqn}中的反常速度项(第一行右手边的第二项)。

式\ref{semi-eqn}刻画了波包运动过程中的信息,为了得到固体的各种性质,我们还需要知道波包在相空间($\bk-\br$空间)的分布情况。在没有磁场的情况下,相空间的态密度是均匀分布的。但是在有磁场的情况下,相空间的态密度则会需要进行修正。这个态密度是
\begin{equation}
D(\br,\bk)=\frac{1}{(2\pi)^d}(1+\frac{e}{\hbar}\boldsymbol{B}\cdot\boldsymbol{\Omega}).\label{semi-dens}
\end{equation}
这里$d$是空间的维度。

式\ref{semi-eqn}和式\ref{semi-dens}是准经典波包理论的重要结果。这两个式子结合在一起,可以预测固体的大量性质,这些预言在实验上获得了巨大的成功。

\section{位移电流}

位移电流是本文的研究重点之一,这一小节我们对位移电流做一个简单的介绍。

位移电流是一种光伏效应,指的是在线性偏振光的照射下,没有空间反演对称性的材料产生自发直流电的现象。目前广泛应用的光电池往往基于$p-n$结,是一种界面效应。与此相反,位移电流来源于体的能带结构,被称为体的光伏效应。

位移电流是一种二阶效应,具体来说,它可以表达为
\begin{equation}
J^\alpha = \sigma^{\alpha\beta\beta}(\omega) E^\beta(\omega) E^\beta(\omega).
\end{equation}
这里的$\alpha$和$\beta$是笛卡尔方向的指标,$\omega$是光场的频率,$\sigma^{\alpha\beta\beta}$被称为位移电导。

位移电流的起源是电子在光场的作用下从价带跃迁到导带时发生的位置变化,位移电导计算表达式为
\begin{equation}
\sigma^{\alpha\beta\beta}=-\frac{2\pi e^3}{\hbar^2}\int \frac{d\bk}{(2\pi)^3}\sum_{n,m}f_{mn}|\berry^\beta_{mn}|^2 R_{nm}^{\alpha\beta} \delta(\omega_{nm}-\omega),
\end{equation}
这里的积分区域为第一布里渊区,$n,m$是能带标号,$f_{mn}=f_m-f_n$是费米分布的差,$\boldsymbol{\berry}_{mn}(\bk)=i \langle u_m(\bk)|\nabla_{\bk}|u_n(\bk) \rangle$是贝里联络的非对角项,$\omega_{nm}=\omega_n-\omega_m$是能带能量的差($\epsilon_n=\hbar\omega_n$), 而$R_{nm}^{\alpha\beta}$具有长度的单位,被称为位移矢量。位移矢量和式\ref{semi-rc}紧密相关,具体形式为
\begin{equation}
R_{nm}^{\alpha\beta}=\partial_\alpha \phi_{mn}^{\beta}-\berry_m^\alpha+\berry_n^\alpha,
\end{equation}
其中$\phi_{mn}^{\beta}$是$\berry_{mn}^{\beta}$的相位。位移矢量的物理阐释就是电子从带$m$跃迁到带$n$发生的位置变化的大小。
\section{布洛赫电子的磁性质}
\subsection{量子振荡}