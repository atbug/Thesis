\chapter{结论}

在本文中,我们研究了如何计算布洛赫电子在电磁场中的响应,以及磁场中近简并能带电子对磁场的响应。

我们首先将Wannier插值方法推广到二阶效应,用以计算位移电流和倍频效应。Wannier插值方法的优势是速度远快于传统的基于波函数的计算方法。我们采用了二阶微扰法解决了对角化Kohn-Sham哈密顿量中出现的相位不定性。通过计算多种二维材料的位移电流和倍频效应,我们证明了这个方法的可靠性。我们将该方法和前人的方法以及紧束缚方法进行了对比,发现我们的方法在效率和准确性上都有优势。

非正交局域基组是很多第一性原理软件包所采用的基组。这些第一性原理软件包高度成熟,可以进行高通量计算。我们将Wannier插值方法推广到了非正交局域基组。薛定谔方程在非正交基组下的表示是广义的本征值方程,我们讨论了广义的本征值方程的微扰论,从而计算出波函数的导数和二阶导数。我们发现非正交局域基组的对称性远好于最大局域化Wannier函数,计算量并没有质的上涨。这个方法可以用于材料基因组的大规模材料探索。

最后,我们研究了磁场中近简并能带的磁响应。通过有效哈密顿量理论,我们推导除了近简并能带的朗道能级的量子化条件。这些条件能够在非简并和完全简并的时候退化到前人的结果。我们还发现这个量子化条件在合适的情况下可以退化到Landau-Zener磁隧穿。一般而言,近简并能带的朗道能级是非简并的,我们研究了什么情况下朗道能级能够出现准简并的现象。准简并的意思是朗道能级之间的能级排斥异乎寻常地弱。我们发现旋转对称性的晶体只需要调节一个参数就可以找到朗道能级的准简并,这一点在量子振荡中表现为周期和振幅的震荡。我们对所有可能的晶体磁群对称性进行了分类,给出了相应对称类下找到一个朗道能级准简并需要的参数数目。

我们的研究从数值上和理论上深化了人们对响应函数的理解,在未来的理论研究和工业应用中可能起到重要的作用。